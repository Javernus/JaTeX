%-------------------------------------------------------------------------------
% LATEX TEMPLATE ARTIKEL
%-------------------------------------------------------------------------------
% Dit template is voor gebruik door studenten van de de bacheloropleiding
% Informatica van de Universiteit van Amsterdam.
% Voor informatie over schrijfvaardigheden, zie
%                               https://practicumav.nl/schrijven/index.html
%
%-------------------------------------------------------------------------------
%	PACKAGES EN DOCUMENT CONFIGURATIE
%-------------------------------------------------------------------------------

\documentclass{jatex-article}
\usepackage[english]{babel}

\usepackage[style=authoryear-comp]{biblatex}
% \addbibresource{references.bib}

%-------------------------------------------------------------------------------
%	GEGEVENS VOOR IN DE TITEL, HEADER EN FOOTER
%-------------------------------------------------------------------------------

% Geef je artikel een logische titel die de inhoud dekt.
\title{Mesh Relay Chat}

\assignment{Individual Report Multimedia}
\assignmenttype{Projectplan}
\documenttype{report}

\authors{Jake Jongejans; (with Anonymous; Merijn Laks; Scott Scherpenzeel)}
\uvanetids{13622552; }

\tutor{Floor van de Leur, Koen van Elsen, Daan Kruis}
\docent{Taco Walstra}

% Vul hier de naam van je tutorgroep, werkgroep, of practicumgroep in.
\group{MRC Alliance (12)}

\course{Multimedia}
\courseid{5062MULT5Y}

\date{\today}

\begin{document}
\justify{}
\maketitle

Mesh Relay Chat, \textit{MRC} for short, is a new chat client that is
disconnected, decentralised, secure and anonymous. It uses a mesh network on
the Meshtastic framework to stay disconnected from the world-wide internet and
instead use the \textit{LoRa} wireless frequencies to communicate. This allows
users to communicate when in relative proximity to one another, but the
technology can also support long range communication over many kilometres. MRC
is decentralised because of the homogeneous nature of the mesh, as each user
has the same function as node on the network. It is anonymous and secure as no
IDing is taking place and all data is encrypted using \texttt{AES256}
encryption, both over the air and on device where messages are stored.

\section*{Project Choices}

As Project Manager, I planned for an MVP in week one, keeping ample time for
set-backs and difficulties along the way. This proved to be a necessity. We
were on track for the front-end of the application, but setting up the mesh
network took longer than expected, and building an interface between Python and
Rust did not prove as easy as one would wish it was.

All in all, we reached our MVP in week two, after which we made the decision to
polish the application rather than add more features. This certainly should be
any teams' first priority.

Along the way, we have iterated over our encryption choices, how we let the
data in our application flow and the

\section*{Wi-Fi, BLE or LoRa}

We have debated three wireless technologies, those being Wi-Fi, Bluetooth Low
Energy and LoRa. There are significant upsides and downsides to using either of
these technologies, as can be seen in Table \ref{wifiblelora}.

\begin{table}[h]
  \centering
  \begin{tabular}{r | c | c | c |}
    \cline{2-4}
                & BLE   & Wi-Fi & LoRa \\ \cline{2-4}
    Range       & Short & Short & Long \\
    Bandwidth   & Low   & High  & Tiny \\
    Power Usage & Low   & High  & Low  \\ \cline{2-4}
  \end{tabular}
  \caption{Comparison between BLE, Wi-Fi and LoRa.}
  \label{wifiblelora}
\end{table}

We are still working out which of these wireless technologies we want to
utilise for MRC, taking into consideration the research we are doing and the
requirements for MRC to function properly.

\section*{Hardware support}

Given the modularity that can be built into MRC, it is possible to create
simple but effective embedded code for any device with the proper capabilities,
like the wireless technology we end up choosing.

Because of its wide availability, cost-effectiveness and ease-of-use, we have
decided to use the ESP32 micro controller, which has bluetooth and wifi built
in. ESP8266 can also be used, as it is the predecessor of the ESP32 and has
almost an identical architecture. Additionally, some boards using the ESP32,
like the TTGO LoRa32 OLED board, can be used for additional features ranging
from better I/O to a built in screen to LoRa capabilities.

\section*{Realism and alternatives}

Given the possible complexity of MRC, we will have to practice care to estimate
the realism of the project. The main cause for complexity will be in
limitations of bandwidth and in the difficulty of programming with the chosen
wireless technology. We are accounting for hidden road blocks in our planning,
leaving ample time for us to handle slow-downs and set-backs.

We aim to have a product by week one, leaving week two for testing, bug fixing,
polishing and if time allows it, creating firmware for more devices so they can
also utilise the MRC network. It also grants us generous time to invest in
writing our individual reports.

Week one focusses on UI and setting up the Mesh Network, of which implementing
the Host of the MRC is part. Then, encryption and add-on modules can be built
to allow extra functionality like sending images, audio.

To ascertain a project, we thought of a backup plan. The backup plan entails
the making of a piece of hardware capable of scrambling audio input from a
microphone. This is done so any internet-connected device does not have access
to the original audio in any way, shape or form.

\section*{Roles}

To ascertain a fluid and effective development sprint, we have decided to
create roles with responsibilities.

The Product Manager keeps track of the state of the project and must make sure
to uphold the progress. They do not deal with the code written, but the product
as it will be used. This role is given to ....

The Code Manager upholds the quality of the code. They manage the git
environment and are the only person allowed to merge Pull Requests. This role
is handed to Jake.

\section*{Planning}

The planning can be found in Figure \ref{planning}.

\begin{figure}[!ht]
  \begin{center}
    \centerline{
      \begin{ganttchart}[
          y unit title=0.5cm,
          y unit chart=0.5cm,
          x unit=0.25cm,
          canvas/.style={fill={rgb,255:red,19;green,38;blue,42},draw={rgb,255:red,141;green,178;blue,185}},
          title label font=\color{rgb,255:red,240;green,255;blue,255},
          title/.style={fill={rgb,255:red,19;green,38;blue,42}},
          link/.style={->,draw={rgb,255:red,224;green,16;blue,34}},
          title label anchor/.style={below=-1.6ex},
          title height=1,
          progress label text={},
          group right shift=0,
          group top shift=.6,
          bar height=0.7,
          bar/.append style={fill={rgb,255:red,95;green,150;blue,125}},
          bar incomplete/.append style={fill={rgb,255:red,89;green,100;blue,112}},
          group height=[.3]{1}{56},

          \setganttlinklabel{f-s}{}
          %labels
          \gantttitle{Project}{56} \\
          \gantttitle{Preparations}{56} \\

          %tasks
          \ganttbar[progress=60]{Audio Research}{1}{56} \\
          \ganttbar[progress=100]{BLE Research}{1}{56} \\
          \ganttbar[progress=60]{Wi-Fi Research}{1}{56} \\
          \ganttbar[progress=100]{LoRa Research}{1}{56} \\
          \ganttbar[progress=90]{Projectplan}{1}{56} \\

          \gantttitle{}{56} \\
          \gantttitle{Week 1}{56} \\
          \gantttitle{Day 1}{8}
          \gantttitle{Day 2}{8}
          \gantttitle{Day 3}{8}
          \gantttitle{Day 4}{8}
          \gantttitle{Day 5}{8}
          \gantttitle{Weekend}{16} \\

          %tasks
          \ganttbar[progress=0,name=meshnw]{Mesh Network}{3}{16} \\
          \ganttbar[progress=0,name=mnwpolish]{Mesh Network Polish}{17}{24} \\
          \ganttbar[progress=0,name=hostimplementation]{Host Implementation}{13}{24} \\
          \ganttbar[progress=0,name=uife]{UI Frontend}{3}{16} \\
          \ganttbar[progress=0,name=uibe]{UI Backend}{21}{32} \\
          \ganttbar[progress=0,name=encryption]{Encryption}{13}{32} \\
          \ganttbar[progress=0]{Modules}{21}{40} \\

          %relations
          \ganttlink[link type=f-s]{meshnw}{mnwpolish}
          \ganttlink{meshnw}{hostimplementation}
          \ganttlink{hostimplementation}{uibe}
          \ganttlink{uife}{uibe}
          \ganttlink{hostimplementation}{encryption}

          \gantttitle{}{56} \\
          \gantttitle{Week 2}{56} \\
          \gantttitle{Day 8}{8}
          \gantttitle{Day 9}{8}
          \gantttitle{Day 10}{8}
          \gantttitle{Day 11}{8}
          \gantttitle{Day 12}{8}
          \gantttitle{Holidays!}{16} \\

          %tasks
          \ganttbar[progress=0]{Polishing}{1}{24} \\
          \ganttbar[progress=0]{Bug fixing}{1}{24} \\
          \ganttbar[progress=0]{More Firmware}{1}{24} \\
          \ganttbar[progress=0]{Testing}{1}{24} \\
          \ganttbar[progress=0]{Reports}{1}{24} \\
          \ganttbar[progress=0,name=poster]{Creating Poster}{21}{32} \\
          \ganttbar[progress=0,name=presentations]{Presenting}{33}{40} \\
          \ganttbar[progress=0,name=freedom]{Freedom}{41}{56} \\

          \ganttlink[link type=f-s]{poster}{presentations}
          \ganttlink[link type=f-s]{presentations}{freedom}
        ]

      \end{ganttchart}
    }
  \end{center}
  \caption{Planning}
  \label{planning}
\end{figure}

\end{document}
