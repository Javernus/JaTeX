%-------------------------------------------------------------------------------
% LATEX TEMPLATE ARTIKEL
%-------------------------------------------------------------------------------
% Dit template is voor gebruik door studenten van de de bacheloropleiding
% Informatica van de Universiteit van Amsterdam.
% Voor informatie over schrijfvaardigheden, zie
%                               https://practicumav.nl/schrijven/index.html
%
%-------------------------------------------------------------------------------
%	PACKAGES EN DOCUMENT CONFIGURATIE
%-------------------------------------------------------------------------------

\documentclass{jatex-article}
\usepackage[english]{babel}

\usepackage[style=authoryear-comp]{biblatex}
% \addbibresource{references.bib}

%-------------------------------------------------------------------------------
%	GEGEVENS VOOR IN DE TITEL, HEADER EN FOOTER
%-------------------------------------------------------------------------------

% Geef je artikel een logische titel die de inhoud dekt.
\title{Mesh Relay Chat}

\assignment{Individual Report Multimedia}
\assignmenttype{Projectplan}
\documenttype{report}

\authors{Jake Jongejans; (with Anonymous; Merijn Laks; Scott Scherpenzeel)}
\uvanetids{13622552; }

\tutor{Floor van de Leur, Koen van Elsen, Daan Kruis}
\docent{Taco Walstra}

% Vul hier de naam van je tutorgroep, werkgroep, of practicumgroep in.
\group{MRC Alliance (12)}

\course{Multimedia}
\courseid{5062MULT5Y}

\date{\today}

\bibliography{references.bib}

\begin{document}
\justify{}
\maketitle

Mesh Relay Chat, \textit{MRC} for short, is a new chat client that is
disconnected, decentralised, secure and anonymous. It uses a mesh network on
the Meshtastic framework to stay disconnected from the world-wide internet and
instead uses the \textit{LoRa} wireless frequencies to communicate. This allows
users to communicate when in relative proximity to one another, but the
technology can also support long range communication over many kilometres. MRC
is decentralised because of the homogeneous nature of the mesh, as each user
has the same function as node on the network. It is anonymous and secure as no
IDing is taking place and all data is encrypted using \texttt{AES256}
encryption, both over the air and on device where messages are stored.

\section*{Project Choices}

As Project Manager, I planned for an MVP in week one, keeping ample time for
set-backs and difficulties along the way. This proved to be a necessity. We
were on track for the front-end of the application, but setting up the mesh
network took longer than expected, and building an interface between Python and
Rust did not prove as easy as one would wish it was.

All in all, we reached our \textit{MVP} in week two, after which we made the
decision to polish the application rather than add more features. This
certainly should be any teams' first priority.

We decided early on to use \texttt{LoRa} as our wireless technology on which
the mesh network functions. This was decided after much research into the
different technologies we could have used. \texttt{Wi-Fi} was another
contender, while \texttt{Bluetooth Low Energy} was quickly deemed insuffient
for our use-case.

Along the way, we have iterated over our encryption choices, ascertaining that
we utilise the highest encryption standards for the safety of the users' chat
messages. A developmental choice later in the project reified this, as we put
our attention to the integrity of our encryption methods.

The additional modules that we envisioned did not get materialised, however, as
we focussed on making a solid application. There is no way to send images or
files. Since text works, though, sending links should not be a problem, which
can link to images and files.

\section*{Minimum Viable Product}

MRC had a single hard requirement: being able to send a message from one app to
another using the mesh network. This required a mesh network to function on the
\texttt{ESP32} boards we had, \textit{Python} code to handle incoming messages
and sending of messages, an interface between \textit{Rust} and Python to add
the messages to the application database, \textit{Tauri} (written in Rust) to
handle the database and updating of the front-end, and the front-end, made
using \textit{SolidJS}, to display the messages and allow for joining groups
and writing messages.

\section*{The Code}

I will not be sharing any code here. I made this decision consciously, as the
code we wrote is highly dependent on the frameworks we utilised. SolidJS
components do not function in Vue or React environments, and Tauri isn't
NodeJS. I will go through a few of the choices and implications of these
choices, however.

We decided to use \textit{SolidJS}, a new and extremely fast front-end
framework that prides itself on its reactivity. SolidJS (but also any other
framework) allowed us to write components to use in our application, which is
integral to the modularity of the code and the maintainability of it. Given its
new status, it is not bound to old standards and legacy code, and it had many
frameworks to use as a frame of reference. This led to a simple yet elegant and
fast framework that allowed us to handle states, resources and effects
intuitively.

Underneath SolidJS, we ran \textit{Tauri}; the back-end. It handles the app,
window creation, and back-end tasks such as communicating with files or
streams. Tauri is written in Rust, a language that handles memory very
efficiently and is able to run exceptionally fast. Tauri implemented
straightforward ways of communicating both from front-end to back-end and vice
versa. It also allowed us to do the many things Rust also allows, using
dependencies.

Lastly, \textit{Meshtastic} was a choice I left up to the experts in the field.
It turned out to run terrifically.

\section*{The Team}

I cannot be more happy with the team. Each of us took a part of the application
and took responsibility for it. Scott and Merijn did quite a bit of research on
which wireless technology we should use. Scott then went on to get the mesh
network working while Merijn worked on the link with Rust. Ilya took it upon
him to work on encryption and learnt to use Rust from scratch. I worked
primarily on getting the front-end looking good and kept everything sailing
smoothly, helping out where necessary. The communication in our team was
top-notch, as we actively discussed decisions and asked questions or help.

I believe everybody has put their all into this project and worked at their
peak, which resulted in a compelling product. None of us slacked off in any way
and the contribution of all of us was integral to the functioning of our
product.

\section*{Bibliography}

\nocite{*}
\printbibliography[heading=none]

\end{document}
